\documentclass{article}

\usepackage[utf8]{inputenc}
\usepackage[T1]{fontenc}

\begin{document}

\title{Manuel d'Utilisation\\
Arène Vectorielle Synchrone{}}
\author{Julien Bissey\\Kelly}
\date{15 avril 2019}

\maketitle

\section{Introduction}
Le but du jeu est de déplacer un vaisseau jusqu'à un objectif pour gagner des points.\\
Ce jeu est divisé en deux parties, Serveur et Client.
Il suffit de faire make à la racine du jeu pour créer les exécutables dans bin.
On peut alors lancer le serveur et les clients :\\
\\
Server <port>\\
java client.Client <ip> <port> <player\_name>\\
\\
On peut également lancer le serveur dans un mode compatible respectant davantage le formulaire de base.\\
\\
Server <port> -comp\\
\\
La principale différence étant que le server au format compatible ne gère pas les collisions élastiques, les obstacles sont donc immobiles.

\section{Serveur}

Une fois le serveur lançé, il n'y a plus besoin de s'en occuper.
Si le server print "please decrease Values.server\_tickrate", cela signifie qu'il n'a pas le temps
de faire tous ses calculs avec son tickrate actuel. Une autre manière de pallier au problème peut être de réduire le nombre d'obstacles ou le nombre de joueurs maximum.
Cela ne devrait pas arriver avec les constantes actuelles.\\

\section{Client}

Une fois le client lancé, il ouvre une fenêtre contenant :\\
\\
- En haut à gauche : l'écran des scores\\
- En bas à gauche : la boîte de texte affichant les divers messages du serveur\\
- La fenêtre principale affichant le jeu.\\
\\
Le vaisseau bleu est celui du joueur, les vaisseaux rouges sont ceux des ennemis. L'objectif est représenté par un disque vert, les obstacles par des cercles blancs.\\
\\
Le joueur dispose de 3 actions :\\
\\
- Flèche haute : accélerer\\
- Flèche gauche : tourner à gauche (babord)\\
- Flèche droite : tourner à droite (tribord)\\

\end{document}
